\documentclass[12pt, letterpaper]{report}
\usepackage[newfloat]{minted}
\usepackage{caption}
\usepackage{tocloft}
\usepackage{acro}

% class `abbrev': abbreviations:
\DeclareAcronym{omr}{
  short = OMR ,
  long  = Optical mark recognition ,
  class = abbrev
}
\DeclareAcronym{oer}{
  short = OER ,
  long  = Open educational resources ,
  class = abbrev
}

\begin{document}
\definecolor{bg}{rgb}{0.95,0.95,0.95}
\frenchspacing
\newenvironment{ex}{\captionsetup{type=listing}}{}
\SetupFloatingEnvironment{listing}{name=Code Sample}
\newcommand{\listcodesamples}{List of Code Samples}
\newlistof{codesample}{mcf}{\listcodesamples}

\newcommand{\codesample}[1]
{
    \captionof{listing}{#1}
    \refstepcounter{codesample}
    \addcontentsline{mcf}{codesample}
    {\protect\numberline{\thecodesample}#1}\par
}

\hyphenation{pro-hib-it-ive-ly Scan-tron ubiq-uit-ous}

\acsetup{first-style=short}

\title{Test Latex Doc}
\author{Ian Sanders}
\date{December 2019}
\maketitle

\begin{abstract}
Lorem ipsum `dolor' sit amet.
\end{abstract}

\tableofcontents
\listofcodesample
\printacronyms[include-classes=abbrev,name=Abbreviations]

\chapter{Introduction}
\section{Background}
Optical mark recognition (\ac{omr}) is a ubiquitous technology whenever large amounts
of controlled physical data needs to be converted to a digital form. This is an
especially common task in the field of education, where the scoring of dozens to
hundreds of multiple choice tests is a regular need. Manual grading of such
examinations is a mundane and time-consuming task that can be easily avoided
with the use of \ac{omr}, as the submission format is completely controlled.
\section{Existing Solutions}
Many of the most popular \ac{omr} solutions presently available are prohibitively
expensive. The most popular product available, sold by Scantron, consists of a
proprietary scanner that is only compatible with Scantron sheets. The scanner
alone can cost several thousand dollars, and the sheets present a continuing
cost for as long as the scanner is in use.

An increasing interest in open educational resources (\ac{oer}) has led to the
development of several free and open source solutions, such as: FormScanner,
queXF, and Auto Multiple Choice. These softwares work extremely well, however,
they have failed to become mainstream due to their increased complexity. Thus,
there exists a need for a simple, free solution that can be implemented with
minimal training or setup time.
\section{Proposed Solution}
A simple, freely available software and mark sheet pair is proposed. By
providing a freely available mark sheet in printable form, setup time is reduced
to the time it takes to download and print the file. In addition, a simple,
easy-to-use software utility that prioritizes reliability will encourage
educators to make the transition from proprietary solutions to an open and free
one. The software will remain simple while still providing a competitive feature
suite and will act as a bridge between raw exam results and more full-featured
exam analysis software.
\subsection{Requirements}
Based on the problem analysis and proposed solution, the final product should:

\begin{enumerate}
  \item Be freely available and open-source
  \item Be as reliable as possible in order to produce fair results
  \item Process files in bulk
  \item Be simple and easy to learn and use
  \item Sort results
  \item Output files that can be input into analysis software
  \item Require minimal setup
  \item Be immediately understood by students taking exams
  \item Provide readily available training resources for educators
  \item Complete processing of results in reasonable amount of time
\end{enumerate}

\chapter{Design of Multiple Choice Sheet}
\section{Requirements}
\section{Design Overview}
\section{Features}
\subsection{Grid System}
\subsection{Corner Marks}

\chapter{Design of MCR Software}
\section{Requirements}
\section{Design Overview}
\section{Features}
\subsection{Automatic Scoring}
\subsection{Rearrangement by Key}
\subsection{Sorting Results}
\subsection{Output Files}
\section{Optical Mark Recognition Process}
\subsection{Image Preprocessing}
\subsection{Corner Detection}
\subsection{Grid Establishment}
\subsection{Mark Reading}


Code block:

\begin{ex}
\codesample{My Python}
\label{codesample:c-code}
\begin{minted}[autogobble, bgcolor=bg]{python}
  def funcname(parameter_list):
    pass
\end{minted}
\end{ex}

example content \ref{codesample:c-code}
\end{document}